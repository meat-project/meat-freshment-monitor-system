\chapter{研究結果}

\section{感測器檢測狀況}
(圖片)\\
使用的三個感測器皆已成功完成線路連接,且能及時讀取氣體濃度,但唯獨 MH-Z19B 重新啟動時需等待其熱機完畢方可使用。

\section{模型時間準確度}
\begin{itemize}
	\item 決策樹
	\begin{figure}[H]
		\centering
		\includegraphics[width=0.8\textwidth]{pic/decisiontree.png}
	\end{figure}
	\item KNN
	\begin{figure}[H]
		\centering
		\includegraphics[width=0.8\textwidth]{pic/knn.png}
	\end{figure}
	\item 單純貝氏
	\begin{figure}[H]
		\centering
		\includegraphics[width=0.8\textwidth]{pic/Bayesian.png}
	\end{figure}
	\item AdaBoost
	\begin{figure}[H]
		\centering
		\includegraphics[width=0.8\textwidth]{pic/Adaboost.png}
	\end{figure}
\end{itemize}

\section{設備外型}
	\subsection{內層殼層}
		\begin{figure}[H]
			\centering
			\includegraphics[width=0.5\textwidth]{pic/box(1).jpg}
		\end{figure}
		\begin{figure}[H]
			\centering
			\includegraphics[width=0.5\textwidth]{pic/box(2).jpg}
		\end{figure}
		\begin{figure}[H]
			\centering
			\includegraphics[width=0.5\textwidth]{pic/box(3).jpg}
		\end{figure}
	\subsection{外層水泥殼層加固}
		後來老師提示後,發現外殼強度不足,所以在外面加了一層水泥
		\begin{figure}[H]
			\centering
			\includegraphics[width=0.5\textwidth]{pic/concrete_box1.jpg}
		\end{figure}
		\begin{figure}[H]
			\centering
			\includegraphics[width=0.5\textwidth]{pic/concrete_box2.jpg}
		\end{figure}
		\begin{figure}[H]
			\centering
			\includegraphics[width=0.5\textwidth]{pic/concrete_box3.jpg}
		\end{figure}