\chapter{研究方法}

\section{目標}
\begin{enumerate}
	\item 該電子設備能夠測出多種腐敗產生的氣體,並且測出濃度。
	\item 該電子設備之靈敏度能高於人體嗅覺可察覺的程度,在人類鼻子無法感測到的濃度之下,
		即可發現該氣體並顯示濃度。
	\item 透過前述腐敗產生氣體的濃度,訓練出一個判斷肉品腐敗的模型,並透過模型判斷狀態,且準確率能高過 8 成。 
	\item 訓練出的判斷模型能夠與設計的硬體裝備整合,變成最終的測量裝置。
\end{enumerate}

\section{方法及步驟}
\begin{enumerate}
	\item 先選定幾個腐敗時具有明顯氣味及濃度變化的化學物質,用不同濃度測試,確認感測器能否在鼻子還無法辨別
		的濃度下,先測出化學物質的濃度。
	\item 並透過數據推論出即將要開始壞掉的食物的數據,並以此定義為即將腐壞。 
	\item 累積數據,並且以數據分析,接著透過機器學習的方式訓練出一個判斷是否腐敗的模型。
	\item 取得肉品產生腐敗時的氣體濃度,且確認該項數據必須能夠在使用者嗅覺能自行決定之前進行判斷,
		能夠避免該項肉品在外觀、氣味正常但已經開始些微腐敗的狀況時被使用者食用。
	\item 作出一個容器用於保存肉品同時收納相關感測器設備並做出區隔,且能夠測量後顯示綠燈(食物完全沒有測出腐敗的跡象),黃燈(食物可能有很低程度損壞),
		紅燈(極建議丟棄)三種標示。 
\end{enumerate}

\section{實驗設備與作業環境}
	\subsection{感測器}
	\begin{enumerate}
		\item MQ-136 \begin{itemize}
			\item 主要晶片:LM393、MQ-136氣體傳感器
			\item 偵測濃度範圍:1 ~ 200ppm 
			\item 工作電壓:DC5V
			\item 具有信號輸出指示
			\item 雙路信號輸出(模擬量輸出及TTL電平輸出)
			\item TTL輸出有效信號為低電平。 (當輸出低電平時信號燈亮,可直接接單晶片)
			\item 模擬量輸出0~5V電壓,濃度越高電壓越高。
			\item 對硫化氫、液化氣,天然氣,城市煤氣,煙霧有較好的靈敏度。
			\item 在本實驗中主要用於硫化氫感測
		\end{itemize}
		\item MQ-137 \begin{itemize}
			\item 主要元件:氨氣氣體感測器
			\item 工作電壓:DC 5V
			\item 特點 \begin{itemize}
				\item 模組帶電平信號輸出,帶報警信號指示燈。
				\item 感測器模組帶類比信號輸出,電壓範圍:0-5V。
				\item 電平信號輸出低電平有效,可驅動PNP三極管,也可接單片機IO口。
				\item 類比量輸出電壓隨濃度增加而增加,濃度越高電壓越高。
				\item 對氨氣、三甲胺、乙醇胺氣體具有很高的靈敏度。
				\item 具有長期的使用壽命和可靠的穩定性。
			\end{itemize}
			\item 在本實驗中用於測量氨氣濃度。
		\end{itemize}
		\item MH-Z19B \begin{itemize}
			\item 檢測氣體:二氧化碳
			\item 工作電壓:4.0~5.5V DC
			\item 平均電流:<60 mA(@5V供電)
			\item 峰值電流:150 mA(@5V供電)
			\item 接口電平:3.3V (兼容5v)
			\item 測量範圍:0~5000ppm
			\item 在本實驗用於測量二氧化碳濃度
		\end{itemize}
	\end{enumerate}
	\subsection{控制板}
	\begin{enumerate}
		\item Arduino UNO R3: 用於連接感測器,並執行感測器的相關讀取程式來取得需要的氣體濃度數據。
	\end{enumerate}
	\subsection{ML模型}
	\begin{enumerate}
		\item 決策樹\\
			決策樹是一個預測模型;他代表的是對象屬性與對象值之間的一種映射關係。樹中每個節點表示
			某個對象,而每個分叉路徑則代表某個可能的屬性值,而每個葉節點則對應從根節點到該葉節點所經歷的路徑
			所表示的對象的值。決策樹僅有單一輸出,若欲有複數輸出,可以建立獨立的決策樹以處理不同輸出。
			數據挖掘中決策樹是一種經常要用到的技術,可以用於分析數據,同樣也可以用來作預測。 
		\item SVM\\
			支援向量機(英語:support vector machine,常簡稱為SVM,又名支援向量網路)是在
			分類與迴歸分析中分析資料的監督式學習模型與相關的學習演算法。給定一組訓練實例,每個訓練實例
			被標記為屬於兩個類別中的一個或另一個,SVM訓練演算法建立一個將新的實例分配給兩個類別之一的模型,
			使其成為非概率二元線性分類器。SVM模型是將實例表示為空間中的點,這樣對映就使得單獨類別的實例
			被儘可能寬的明顯的間隔分開。然後,將新的實例對映到同一空間,並基於它們落在間隔的哪一側來預測
			所屬類別。
		\item 單純貝氏\\
			單純貝氏是一種構建分類器的簡單方法。該分類器模型會給問題實例分配用特徵值表示的類標籤,類標籤取
			自有限集合。它不是訓練這種分類器的單一演算法,而是一系列基於相同原理的演算法:所有單純貝氏分類器
			都假定樣本每個特徵與其他特徵都不相關。舉個例子,如果一種水果其具有紅,圓,直徑大概3英寸等特徵,
			該水果可以被判定為是蘋果。儘管這些特徵相互依賴或者有些特徵由其他特徵決定,然而單純貝氏分類器
			認為這些屬性在判定該水果是否為蘋果的機率分布上獨立的。對於某些類型的機率模型,在監督式學習的樣本
			集中能取得得非常好的分類效果。在許多實際應用中,單純貝氏模型參數估計使用最大概似估計方法;
			換而言之,在不用到貝氏機率或者任何貝葉斯模型的情況下,單純貝氏模型也能奏效。
	\end{enumerate}