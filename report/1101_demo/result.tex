\chapter{研究結果}

\subsection{第一次進度報告進展}
\begin{enumerate}
	\item 採用樹莓派作為控製板 \begin{itemize}
		\item 配置好樹莓派伺服器。
		\item 以樹莓派作為裝置運算與控制的主體,並設定「通訊埠轉發」來達到組員可隨時以 ssh 遠端控制,
			測試控制碼。 
	\end{itemize}
	\item 線路連接 \begin{itemize}
		\item 焊接感測器與排母
		\item 使其變為跳線或者杜邦線的規格,能與樹莓派線路界面連接。
	\end{itemize}
	\item 裝置設計:建立初步裝置設計圖
	\item 控制碼及模型設計:利用 python 及其支援的 module ,熟悉並寫出簡單的測試碼
	\item 遭遇的問題 \begin{itemize}
		\item 感測器與機板的連接問題
		\item 供電問題
		\item 讀取資料的程式
		\item 絕緣密閉
		\item 系統穩定度
		\item 詳細定義食物的狀態( ex. 如何為腐敗?),並使之量化,能放入實驗設計
	\end{itemize}
	\item 短期改善與預計達成目標 \begin{itemize}
		\item 欲先解決感測器連接問題(目前考慮重買器材?)
		\item 要能夠使裝置能夠長時間運行,監測肉類,維持穩定
		\item 設計肉類判別流程,該流程需可以明確判定肉類是否已腐敗
	\end{itemize}
\end{enumerate}

\section{第二次進度報告進展}
\begin{enumerate}
	\item 絕緣問題解決(如附圖)
	\item 短期改善與預計達成目標 \begin{itemize}
		\item 讀取資料的程式問題
		\item 設計實驗流程
		\item 如何訓練判斷的模型
	\end{itemize}
\end{enumerate}

\section{第三次進度報告進展}
\begin{enumerate}
	\item 確認所有感測器皆可正常運作 \begin{itemize}
		\item 因為樹莓派的 python module 對於感測器支援較少,因此進度上遇到困難
		\item 改用 arduino uno 作為我們接下來實驗用到的板子。
	\end{itemize}
	\item 改變控制板 \begin{itemize}
		\item 讀取資料的程式問題
		\item 設計實驗流程、訓練判斷的模型
		\item arduino uno 與感測器的整合
	\end{itemize}
	\item 短期改善與預計達成目標
\end{enumerate}