\chapter{文獻回顧}

\section{相關化學物質}
肉類腐敗散發臭氣的多種的化學物質:硫化氫(H2S)、氨氣 (NH3)、硫醇等等的化合物。
(取自參考文獻 2 - GAS 優良肉品網) 
\begin{enumerate}
	\item 硫化氫 \begin{itemize}
		\item 無機化合物,化學式為H2S。正常是無色、易燃的酸性氣體,也是一種氧族元素的氫化物。
		\item 是急性劇毒物質,具有臭雞蛋味,吸入少量高濃度硫化氫可於短時間內致命。
			低濃度的硫化氫對眼、呼吸系統及中樞神經都有影響。它有毒,有腐蝕性,還可以被燃燒。
		\item 蛋白質在酵素分解下,會產生硫化氫
	\end{itemize}
	\item 氨氣 \begin{itemize}
		\item 無色氣體,有強烈刺激氣味(尿味),極易溶於水。
		\item 氨對地球上的生物相當重要,是所有食物和肥料的重要成分。
			氨也是很多藥物和商業清潔用品直接或間接的組成部分,具有腐蝕性等性質。
	\end{itemize}
	\item 硫醇 \begin{itemize}
		\item 除甲硫醇在室溫下為氣體外,其他硫醇均為液體或固體。
		\item 硫醇與二硫化物形成的氧還共軛對是生物體內的常見機制,如半胱氨酸-胱氨酸還氧對。
			生成的二硫化物中的二硫鍵在維持蛋白質空間結構方面有重要作用。
	\end{itemize}
\end{enumerate}

\section{電子鼻}
Fast GC Analyzer 電子鼻 (簡稱 zNose® ,Electronic Sensor Technology),
使用 Tenax 為吸附劑,待吸附氣分子後,利用熱脫附裝置,使氣味分子快速進入層析系統,
於 10-20 秒內完成分離後,以表面聲波共振(surface acoustic wave resonator) 偵測,
記錄各分離物種之滯留時間與信號強度,繪製嗅覺影像圖,與已知的氣味嗅覺影像圖資料比對,
得知氣味的種類。 (取自參考文獻 1 - 異味氣體之偵測) 

\section{Alpla MOS}
Alpha MOS 由 6 個氣體感測器陣列所組成的 RQBOX 模組氣體監測系統,具有無線訊號傳輸裝置,
模組內的無線傳輸裝置可提供即時的遠端訊號傳輸,實際測試的結果顯示,其傳輸距離可達三百多公尺遠。
藉由連接有訊號接收器的電腦,可以提供遠端遙控即時空氣品質監測的功能。
(取自參考文獻 1 - 異味氣體之偵測) 

\section{感知比較實驗}
進行人的嗅覺系統對於臭氣的感知比較實驗(三點比較式嗅袋法)-人體嗅覺是否能察覺有害氣體。
(取自參考文獻 1 - 異味氣體之偵測) 

\section{氣味感測器實例應用}
(取自參考文獻 1 - 異味氣體之偵測)\\
	\subsection{油管漏油事件}
	輸油管不論是遭受意外或人為蓄意破壞,漏油所影響的面積和環境污染程度最為嚴重。
	應用電子鼻分析結果。顯示整合現場快速判定未知樣品之電子鼻分析技術,協助取得代表性樣品,
	藉由 GC-MS 確認分析,同時達到緊急應變處置,降低災害至最低。
	\subsection{辦公大樓室內空氣異味事件}
	2002 年 11 月前往內疑有異味且地面有異物之辦公大樓,進行採樣檢測以了解可能原因。
	\subsection{羊乳摻牛乳事件}
	使用儀器分析羊乳中是否有牛乳摻假,多利用羊、牛乳中之脂肪酸或蛋白質 (酪蛋白或乳清蛋白) 組分之差異。
	使用電子鼻具有檢測時間短、檢測方法簡單、操作方法容易等優點。

\section{氣味成分分類}
(取自參考文獻 3 - 肉類食品的電子鼻辨識)
\begin{itemize}
	\item PCA method consists to choosing the principal components axes. 
	These axes are used to obtain a quite precise summary of the information 
	contained in the database. The graphics are constructed to give a meaning for 
	the new variables and provide an evaluation of the quality of this summary. 
	This method is used to classify data in groups. 
	\item DFA is a method of data analysis aiming to discriminate m groups of 
	previously defined individuals, described by p quantitative variables. We will 
	thus seek linear combinations of the p initial variables (discriminate axis) 
	leading to the best separation between groups. This allows, among other things, 
	to describe the differences between these groups. A statistic software XLSTAT 
	is used to implement the PCA and DFA methods. 
\end{itemize}