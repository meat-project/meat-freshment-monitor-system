\chapter*{摘要}
\addcontentsline{toc}{chapter}{摘要}

本次專題預期將製作一可收納食物,並在食物腐敗、散發不良氣味之前,能夠感測出來
並即時通知的裝置,藉此避免造成使用者在不清楚食物已經腐敗的狀況下進行食用。
首先,不同的食物腐敗時散發出的化學成分不盡相同,因此我們打算先以偵測肉類食品
為出發點進行實作,肉類食品之所以放久了會腐敗產生惡臭味,是因為肉品中的微生物在
生長時所產生化學物質,如假單胞菌分解蛋白質,所產生硫化物,如硫化氫($H_2S$)、氨氣($NH_3$)、
硫醇等等的化合物,而這類分解作用,也就是俗稱的「腐敗」,藉由能夠偵測先前提到的腐敗時
所產生的氣體的氣體感測元件收集到的數據,來判斷該肉品是否可以繼續食用,順利的話
將增加不同種類的食物判斷標準,且考慮更多不同食物腐敗時可能產生的其他變化特性。
例如:pH值、水份含量等等來判斷是否繼續食用。 