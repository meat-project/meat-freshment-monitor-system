\chapter{結論}

本次專題出發點的概念看似非常簡單,但是在實際執行方面還是遇到的不少的問題,例如:感測器的資料讀取、
系統的整合、「腐敗」狀態的判定、實驗肉品種類的選擇等等,而在腐敗的判定則是困擾我們最久的障礙,
因為一直無法找到客觀的數據佐證,幸虧在指導教授的指導和我們的討論下,選擇以放置時間作為判別標準,
且給出我們對於該時間狀態下的食用建議,但實際是否要拿來煮食則是交由使用者自行判斷。距離我們實際理想的目標也已完成約七、八成,
之後可能將在「裝置的外盒設計、材質」、「對使用者進行狀態的提醒」這兩方面進行改善,增加使用體驗上的好感度。\\

之後若有餘力,想繼續擴展本次專題最初的理念 — 對所有食物的新鮮度判定,例如在判定的對象範圍增加對海鮮、蔬菜等等,
以及增加對新鮮度狀態更多的種類及資訊,讓使用者對食品保存的狀況有更近一步的了解。最後,為了讓使用者能更即時、方便地得知資訊,
也有想過開發能夠和裝置連線,透過手持裝置及可一目了然的手機應用程式,讓我們的專題概念能夠以最簡潔俐落、但又不失便利性和功能性的姿態的目標持續進步。