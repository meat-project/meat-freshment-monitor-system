\chapter{簡介}

\section{新鮮程度判定}
平常在廚房中,如果是明顯腐敗有異味,或是剛買回來的新鮮肉類食材,都能明顯決定是否要拿來吃。
但有時仍會疑惑,有些放了一段時間,可能外表還保持乾淨,但是觸感有點不太一樣。或者看似有很小部份異常,
不確定是否損壞,整個食物丟掉又覺得很可惜時,就很希望有存在一個實際能判斷的儀器或方法來解決問題。

\section{腐敗的特徵}
腐敗狀況的判定大致可以藉由觀察顏色變深、表面黏稠度、彈性、產生的異味氣體的程度決定。
色澤的部分,新鮮的肉表面有光澤,多呈紅色或淡紅色,且顏色均勻。但隨著存放時間的延長,由於肌紅蛋白被氧化,肉色會逐漸變成紅褐色。顏色越深,可食性越低。而當肉表面變成灰色或灰綠色,甚至出現白色或黑色斑點時,說明微生物已經產生大量的代謝產物,這樣的肉就可能已經不適合食用了。
黏稠度方面,新鮮的肉外表微乾或濕潤,切面稍潮濕,用手摸有油質感,但不發黏。而肉變質以後,由於微生物大量滋生,會產生黏性代謝產物,造成肉表面發黏,甚至出現拉絲。肉類表面發黏是腐敗開始的特徵。
至於彈性,新鮮的肉質地緊密且富有彈性,用手指按壓凹陷後會立即復原。但存放越久,肉裡面的蛋白質、脂肪會逐漸被酶分解,肌纖維被破壞,所以肉會失去原有的彈性,手指壓後的凹陷不僅不能完全復原,甚至會留有痕跡。
最後在氣味方面,新鮮肉具有正常的肉味,而變質的肉由於蛋白質、脂肪、碳水化合物被微生物分解,會產生各種胺類、、酸類、酮類等物質,進而產生明顯的腐臭味。

\section{預計研究問題}
\begin{itemize}
	\item 確認「機器測試」是否比「人的鼻子」靈敏或者準確?
		能否實際拿來作為分析食物新鮮度的輔助器具?
	\item 定義與量化本次探討主題「腐敗」的判定,以方便日後實驗描述。
	\item 由於腐敗的特徵眾多,因此我們選擇以最為客觀且容易量化的「氣味」層面著手研究,嘗試檢測與肉類腐敗相關的氣體濃度,並判斷其關聯程度的高低。
		最後總和出一個兼具有效率(成本低,時間少)且精準的判斷模型。
	\item 透過模型設計出相關的硬體設備。
\end{itemize}