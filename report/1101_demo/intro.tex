\chapter{簡介}

\section{研究動機}
平常在廚房中,如果是明顯腐敗有異味,或是剛買回來的新鮮食材,都能明顯決定是否要拿來吃。
但有時仍會疑惑,有些食材外表乾淨,但是觸感有點不太一樣。或者看似有很小部份異常,
不確定是否損壞,整個食物丟掉又覺得很可惜時,就很希望有個便利的儀器,能夠隨時一測馬上得到答案。
幫助自己做決定。因此想要嘗試是否能夠依據所學與目前找的到的前人研究,自製一個兼具偵測與判斷的機器。

\section{研究問題}
\begin{itemize}
	\item 測試「機器測試」是否比「人的鼻子」靈敏或者準確?
		能否拿來作為分析食物留下與否的輔助器具?
	\item 定義與量化本次探討主題「腐敗」的判定,以方便日後實驗描述。
	\item 探討不同變因(不同化學成份,酸鹼值,濕度等等)與肉類腐敗判斷的準確性。
		並總和出一個兼具有效率(成本低,時間少)且精準的判斷模型。
	\item 透過模型設計出相關的硬體設備。
\end{itemize}