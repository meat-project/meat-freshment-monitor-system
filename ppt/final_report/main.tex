%!TEX program = xelatex
\documentclass[a4paper, 12pt]{article}
\usepackage{xeCJK, geometry, fontspec}
%-----------------------------------------------------
\setCJKmainfont{教育部標準楷書}
\setmainfont{Times New Roman}
\linespread{1.5}
%-----------------------------------------------------
\title{
	{\bf 肉類食品新鮮度監測系統}\\
	{\bf Meat Freshment Monitoring System}\\[2cm]
	國立中山大學資訊工程學系\\
	110 學年度大學部專題製作競賽\\[2cm]
}
\author{
	Arthor: 李銘庭 (B073040005), 伍建瑋 (B073040008), 蔡孟師 (B073040019)\\[1cm]
	Advisor: 賴威光
}
\date{}
\begin{document}
\section{摘要}
本次專題預期將製作一可收納食物,並在食物腐敗、散發不良氣味之前,能夠感測出來並即時通知的裝置,藉此避免造成使用者在不清楚食物已經腐敗的狀況下進行食用。首先,不同的食物腐敗時散發出的化學成分不盡相同,因此我們打算先以偵測肉類食品為出發點進行實作,肉類食品之所以放久了會腐敗產生惡臭味,是因為肉品中的微生物在生長時所產生化學物質,如假單胞菌分解蛋白質,所產生硫化物,如硫化氫(H2S)、氨氣(NH3)、硫醇等等的化合物,而這類分解作用,也就是俗稱的「腐敗」,藉由能夠偵測先前提到的腐敗時所產生的氣體的氣體感測元件收集到的數據,來判斷該肉品是否可以繼續食用,順利的話預期將增加不同種類的食物判斷標準,且考慮更多不同食物腐敗時可能產生的其他變化特性,例如:pH值、水份含量等等來判斷是否繼續食用。 
\end{document}